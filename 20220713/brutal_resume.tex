\documentclass{resume} % Use the custom resume.cls style

\usepackage[left=0.4in,top=0.2in,right=0.4in,bottom=0.2in]{geometry} % Document margins
\newcommand{\tab}[1]{\hspace{.2667\textwidth}\rlap{#1}} 
\newcommand{\itab}[1]{\hspace{0em}\rlap{#1}}
\linespread{0,7}

\hypersetup{
    colorlinks=false,
    linkcolor=blue,
    filecolor=magenta,      
    urlcolor=blue, %cyan
}

% \newcommand{\checkeditem}{\item[\uncheckmark]}

\name{Kirill Bas} % Your name

\address{+7-(983)-410-62-05 \\ \href{mailto:baskirill.an@gmail.com}{baskirill.an@gmail.com} \\ \href{https://t.me/baskirill}{t.me/eertree\_work}} %\href{https://github.com/bas-kirill}{github.com/bas-kirill} \\   % Your phone number, email, linkedin, and (optional) website

\begin{document}

% %----------------------------------------------------------------------------------------
% %	OBJECTIVE
% %----------------------------------------------------------------------------------------

% \begin{rSection}{OBJECTIVE}

% {Software Engineer with 2+ years of experience in XXX, seeking full-time XXX roles.}


% \end{rSection}
%----------------------------------------------------------------------------------------
%	EDUCATION SECTION
%----------------------------------------------------------------------------------------

\begin{rSection}{Education}

{\bf Saint Petersburg State University}  % \hfill {Sep 2020 - Jun 2024 (expected)}
\begin{itemize}
    \item Bachelor in Mathematics and Computer Science %, \textbf{GPA: 4.7 / 5.0}
    \item Relevant courses: Calculus, Linear Algebra, Programming in C and C++, Intro to CS
\end{itemize}
% Relevant Coursework: A, B, C, and D.

% {\bf Advanced Educational and Scientific Center of Moscow State University} \hfill {Sep 2018 - Jun 2020}
% \begin{itemize}
%     \item Top high school according to RAEX rating
% \end{itemize}

% {\bf Yandex School of Data Analysis} \hfill {Sep 2021 - Present}
% \begin{itemize}
%     \item Highly competitive Computer Science study program in Russian Federation
%     \item \textbf{Relevant courses}: Algorithms and Data Structures in external memory, Databases
% \end{itemize}

%Minor in Linguistics \smallskip \\
%Member of Eta Kappa Nu \\
%Member of Upsilon Pi Epsilon \\
% Relevant courses: Percolations in graphs, Chromatic number of a Kneser graph and \\  its surroundings
% Relevant courses: Adavanced Algorithms and Data Structures


\end{rSection}

%----------------------------------------------------------------------------------------
%	WORK EXPERIENCE SECTION
%----------------------------------------------------------------------------------------

\begin{rSection}{EXPERIENCE}

% \textbf{DTL Private Ltd, Hedge Fund, Software Engineer Intern} \hfill Summer 2022
% \begin{itemize}
%     \item DTL is a Singapore-based investment fund that has more than \$903 million assets under management 
%     \item Design and implement robust internal systems/capabilities to boost the productivity and enhance the trading capabilities of the company
% \end{itemize}

\textbf{Yandex LLC, FinTech, Software Engineer} \hfill Summer 2021 - Summer 2022
\begin{itemize}
    \item Developed, launched and maintained multiple microservices that simplifies communication between buisness and customers, allows to track the status of order readiness
    \item Provisioned an easily manageable hybrid infrastructure (Yandex Cloud) utilizing IaC tool Terraform
    \item Installed and configured system consisting of Kibana, ElasticsearchFinTech

    % \item Installed and configured a dashboards in Grafana for entire core banking department
    % \item Worked on Kotlin framework for fast creating a new microservice in 1 hour: generating REST API controllers by OpenAPI and allocating resources by Terraform
    % \item Used Docker and Kubernetes for managing microservices
    % \item Added new important metrics via Prometheus that shows in Kibana
    % \item Took responsibility for mentoring newbies
\end{itemize}

\textbf{Yandex LLC, ML infra, Software Engineer Intern} \hfill Spring 2021 
\begin{itemize}
    \item Building realtime streaming embedding pipeline which improves recommendation quality by 3.9% and advertising campaigns income by 6.3%
    \item Realtime pipeline consumes user events (3 TB / day), pushes them into sharded persistent queues, aggregates item history on the fly, computes embeddings using matrix factorization methods (ALS and friends)
    \item Configuring monitoring via in-house technology Solomon (like Grafana) which improves observability by 2% recomendations
\end{itemize}
\textbf{Teacher of Algorithms and Data Structures} \hfill Fall 2020 - Spring 2021
\begin{itemize}
    \item Teaching the most used algorithms and data structures in programming competitions % in software development
\end{itemize}
% Teaching for first-year students the most used algorithms and data structures in competitions and their application in software development
% \begin{itemize}
    % \itemsep -3pt {} 
    %  \item Achieved X\% growth for XYZ using A, B, and C skills.
    %  \item Led XYZ which led to X\% of improvement in ABC
    % \item Developer XYZ that did A, B, and C using X, Y, and Z. 
%  \end{itemize}
 
% \textbf{Role Name} \hfill Jan 2017 - Jan 2019\\
% Company Name \hfill \textit{San Francisco, CA}
%  \begin{itemize}
%     \itemsep -3pt {} 
%      \item Achieved X\% growth for XYZ using A, B, and C skills.
%      \item Led XYZ which led to X\% of improvement in ABC
%     \item Developer XYZ that did A, B, and C using X, Y, and Z. 
%  \end{itemize}

\end{rSection} 
%----------------------------------------------------------------------------------------

\begin{rSection}{ACHIEVEMENTS} 
\begin{itemize}
    \item Participated in final stage of Russian National Olympiad in Programming: \textbf{146th} of over 50000 participants and 3rd degree award (\textbf{2019}), \textbf{99 percentile} % , 140th ... (2018), ...
    \item Became an \textbf{Expert} at \href{http://codeforces.com/profile/n2k}{\underline{Codeforces}}, \textbf{95 percentile}
    % \item \textbf{2019:} 3rd degree award in the national stage of the Russian National Olympiad in Programming, 146th of over 50000 participants
    \item Participated in final stage of Russian National Team Olympiad in Programming: \textbf{140th} of over 1000 teams (\textbf{2017}), \textbf{157th} of over 1000 teams (\textbf{2016}), \textbf{90 percentile}
    % \item \textbf{2016, 2017:} Participant of the national stage of the Russian National Team Olympiad in Programming, 140th and 157th accordingly of over 1000 teams 
    % \item \textbf{2016:} Participant of the national stage of the Russian National Team Olympiad in Programming, 140th of over 1000 teams
    % \item Took online courses at stepik.org and cousera.org: "Deep Learning School", "Programming in C++", ”Build a Modern Computer from First Principles: From Nand to Tetris”, ”Algorithms and Data Structures”, ”Introduction in Machine Learning”
    % on operating systems, advanced C++, algorithms and data structures, machine and deep learning
\end{itemize}
\end{rSection}

%----------------------------------------------------------------------------------------
% PROJECTS	
%----------------------------------------------------------------------------------------
\begin{rSection}{PROJECTS}
% \vspace{-1.25em}
% (\underline{\href{https://github.com/bas-kirill/rno-statistics}{code}})
\textbf{Russian National Olympiad Statistics, Web Application}  %\hfill {Aug 2020 - Present} 
\begin{itemize}
    % \item A site with the distribution of participants who passed to the All-Russian level by year from each region of the country 
    \item Created the website layout and prepared the data about the participants of the Olympiad to be used for interactive statistics (map charts, bar histograms, line charts)
    % \item Created the website layout and prepared the data to be used for interactive statistics (map charts, bar histograms, line charts) of participants' distribution who passed to the All-Russian level of Russian National Olympiad in Programming by year from each region of country
    % the data about the participants of the Olympiad to be used for interactive statistics
    % \item  Created the website layout and prepared the data to be used for interactive statistics (map charts, bar histograms, line charts) of participants' distribution who passed to the All-Russian level of Russian National Olympiad in Programming by year from each region of country
    \item Working on the interaction of the front-end interface of the website and the \textbf{Google Charts API} to display user-friendly interactive diagrams
    % \item Working on the interaction between frontend and the Google Charts \textbf{API} to
    % about the final stage of the All-Russian Olympiad
    % \item Collected information on all the final stages of the all-Russian Olympiad
    % \item \textbf{Github link:}  \href{https://github.com/bas-kirill/rno-statistics}{github.com/bas-kirill/rno-statistics}
\end{itemize}

% \item \textbf{Wi-Fi Analyzer, Android / Java Application} \hfill {Sep 2020 - Oct 2020}
% \begin{itemize}
%     % \item Created an application for the Android platform that collects all information about the network and the parameters of connecting the phone to it:
    %     \begin{itemize}
    %         \item Internet access speed using speed test
    %         \item Response time to Internet resources
    %         \item Wi-Fi connection status(RSSI, SNR, MAC)
    %         \item User location for creating a connection map
    %         \item Client device data (manufacturer, OS, versions, chipsets)
    %     \end{itemize}
%    \item Created an application for the Android platform that collects information about the \textbf{network} (Internet access speed, RSSI, SNR, MAC, user location, etc)  and the parameters of connecting the phone to it
%    \item Designed \textbf{UI} programmatically that facilitated dependency injection and reusability of the Android views
%\end{itemize}

% \item \textbf{Polynomials, C++ Application (\underline{\href{https://github.com/bas-kirill/aesc-msu-polynomials}{code}})} \hfill{Oct 2019 - Nov 2019}
%\item \textbf{Polynomials, C++ Application} \\ %\hfill{Oct 2019 - Nov 2019} 
% \begin{itemize}
%** Developed console application that allows perform mathematical actions with them: arithmetic operations, Newton's method for finding root and factorization by prime factors
    % Developed a console application that allows manipulating(arithmetic operations, finding the root using Newton's method, factorization) and storing in database polynomials.
        % \begin{itemize}
        %     \item Database of used polynomials: adding, deleting, and displaying a polynomial
        %     \item Arithmetic operations: addition, subtraction, multiplication, division in optimal time
        %     \item Finding the root of a polynomial using Newton's method
        %     \item Factoring a polynomial
        % \end{itemize}
    
    % \item \textbf{Github link:} \href{https://github.com/bas-kirill/aesc-msu-polynomials}{github.com/bas-kirill/aesc-msu-polynomials/}
% \end{itemize}

% \item \textbf{Calendar, C++ Application (\underline{\href{https://github.com/bas-kirill/aesc-msu-calendar}{code}})} \hfill {Feb 2019 - Feb 2019}
% \begin{itemize}
%     \item Built a graphical application on \textbf{Qt} that displays a calendar for a specific year \& month and allows users to add notes about the date
%     % \item Built a calendar program that shows a user-selected calendar and takes notes on it
%     % \item \textbf{Github link:} \href{https://github.com/bas-kirill/aesc-msu-calendar}{github.com/bas-kirill/aesc-msu-calendar}
% \end{itemize}

% \item \textbf{Graphics, C++ Application (\underline{\href{https://github.com/bas-kirill/aesc-msu-graphics}{code}})} \hfill {Jan 2019 - Jan 2019}
% \item \textbf{Graphics, C++ Application} \hfill {Jan 2019 - Jan 2019} \\
% \begin{itemize}
% ** Implemented an approximate algorithm that finds an area using an integral and an inclusion-exclusion formula applying SFML  library
    
    % Realized graphical application and approximate algorithm that finds an area using an integral and an inclusion-exclusion formula. Learned for it \textbf{SFML} in 1 week
    % \item Learned \textbf{SFML} in 1 week and implemented a graphical application with an approximate algorithm that finds an area using an integral and an inclusion-exclusion formula
    % \item Implemented a graphical application that finds the area of a shape limited by functions
    % \item Learned \textbf{SFML} in 1 week and implemented an approximate algorithm for finding an area using an integral and an inclusion-exclusion formula
    % \item \textbf{Github link:} \href{https://github.com/bas-kirill/aesc-msu-graphics}{github.com/bas-kirill/aesc-msu-graphics}
% \end{itemize}

\item \textbf{Encryptor, Android / Java Application}   %\hfill {Nov 2016 - Mar 2017}
\begin{itemize}
    \item Built application that encrypts text messages using chosen algorithm and shows the encryption steps
    \item Implemented various ciphers using Java \textbf{cryptographic} libraries (e.g., Caesar, Vigenere, Atbash, ROTX, Polybius, RSA)
    % Morse, 
    % \item Got skills in working with cryptographic libraries in Java
\end{itemize}
\end{rSection} 

%----------------------------------------------------------------------------------------

%----------------------------------------------------------------------------------------
% TECHINICAL STRENGTHS	
%----------------------------------------------------------------------------------------
% \begin{rSection}{SKILLS}

% \begin{tabular}{ @{} >{\bfseries}l @{\hspace{6ex}} l }
% Programming Languages & C/C++, Python, Java, SQL, HTML/CSS, Bash
% \\
% Technologies & Qt, SFML, Windows, Linux\\
% Languages & native Russian, intermidiate English, beginner Spanish\\
% \end{tabular}\\
% \end{rSection}

\begin{rSection}{Additional education}
% {\bf Summer School: Combinatorics and Algorithms} \hfill {Aug 2020 - Sep 2020} \\
%Relevant courses: Percolations in graphs, Chromatic number of a Kneser graph

{\bf Advanced Educational and Scientific Center of Moscow State University} \hfill {Sep 2018 - Jun 2020}
\begin{itemize}
    \item \textbf{Top \#1} high school accroding to RAEX rating
\end{itemize}

% {\bf IT School Samsung: Android Development} \hfill {Sep 2016 - Jun 2017}
% \begin{itemize}
    %\item Advanced Training Program in Software Engineering
% \end{itemize}
% \end{rSection}

% \begin{rSection}{Volunteer experience} 
% \begin{itemize}
    % \item ACM ICPC World Finals 2021: World Programming Championship
% \end{itemize}
\end{rSection}

%----------------------------------------------------------------------------------------
%----------------------------------------------------------------------------------------
% \begin{rSection}{COURSES} 
% \begin{itemize}
    
% \end{itemize}
% \end{rSection}

%----------------------------------------------------------------------------------------

% \begin{rSection}{HOBBIES} 
% \begin{itemize}
%     \item Teaching students algorithms and data structures
%     \item Organization of student events
%     \item Studying two foreign languages: English, Spanish
% \end{itemize}
% \end{rSection}


\end{document}

